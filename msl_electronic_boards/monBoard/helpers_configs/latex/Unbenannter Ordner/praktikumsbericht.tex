\documentclass[
	11pt,								% 11 Punkt Schrift verwenden (auch 10pt, 12pt moeglich)
	a4paper,						% Dokumentgroesse A4
	oneside,						% einseitiger Druck (auch twoside moeglich)
	titlepage,					% Titelblatt generieren
	headsepline,				% Kopfzeile duch Linie vom text getrennt
	DIV13,							% Groesse des Satzspiegels
	abstracton,	 				% zeigt die Abstractueberschrift (abstracton oder abtractoff)
	BCOR0cm,						% Groesse des Bindungsrandes (z. B.: BCOR2.5cm)
	bibliography=totoc, % bibliography is added to the table of contents
]{scrreprt}							% Dokumentenklasse (KomaScript - Report)
%]{report}							% Dokumentenklasse (Native Latex - Report)

% MATH, FORMULAE AND SIGNS PACKAGES
\usepackage{amsmath}								% besser als standard math
\usepackage{amssymb}

% LANGUAGE PACKAGES
%\usepackage[german]{				% Anpassungen ensprechend der Sprache
%\usepackage[latin1]{inputenc}
\usepackage[ngerman]{babel}
\usepackage[T1]{fontenc}

\usepackage[utf8]{inputenc}				% direkte Eingabe von Umlauten
\usepackage{csquotes}							% Anfuehrungszeichen entsprechend der Sprache

% GRAPHIC PACKAGES
\usepackage{graphicx}							 % Packet zu Einbindung von Graphiken
\usepackage{subfig}									% fuer die Erstellung von Unterabbildungen

% LAYOUT PACKAGES
\usepackage{fancyhdr}
\usepackage{parskip}								% besseres Absatzlayout (Leerzeile statt Einrückung)
\usepackage[official,right]{eurosym}% für ein \euro{} Symbol
\usepackage{hyperref}							% F\"ur Hyperlinks
\hypersetup{
	hypertexnames=true,
	unicode = {true},
	colorlinks = {true},
	pdftitle = {Entwicklung eines Systems ...},
	pdfauthor = {Christian Schaub},
	pdfkeywords = {keywords},
	pdfborder = {0 0 0},
	linkcolor = black,
	citecolor = black,
	urlcolor = black,
	breaklinks = true,
}
\usepackage{url}										% fuer URLs als Literaturverzeichnis
\usepackage{color}
\usepackage{setspace}
\usepackage{rotfloat}
%\usepackage{verbatim}								% multiline comments
\usepackage{nextpage}
\usepackage[textsize=scriptsize]{todonotes}
\usepackage[numbers]{natbib}


%\usepackage[
%	block=nbpar,
%	citestyle=numeric-comp,
%	bibstyle=stopfer,
%	maxbibnames=10
%]{biblatex}
%\addbibresource{bibtex_database.bib}

% OPTIONS, DEFINITIONS
\usetikzlibrary{arrows,automata,shapes.multipart,trees}
\onehalfspacing							% anderthalbzeilig (oder auch \doublespace)
\pagestyle{headings}				% Überschrift in der Kopfzeile und Seitenzahlen
\setcapindent{0mm}
\addtokomafont{caption}{\small}

% OWN COMMANDS

% Adding lines above and below the chapter head
\newcommand*{\ORIGchapterheadstartvskip}{}
\let\ORIGchapterheadstartvskip=\chapterheadstartvskip
\renewcommand*{\chapterheadstartvskip}{
	 \ORIGchapterheadstartvskip{
			 \setlength{\parskip}{-40pt}
			 \noindent\rule{.3\textwidth}{3pt}\rule[2.5pt]{.7\linewidth}{.5pt}\par
	 }
}

\newcommand*{\ORIGchapterheadendvskip}{}
\let\ORIGchapterheadendvskip=\chapterheadendvskip
\renewcommand*{\chapterheadendvskip}{
	 {
			 \setlength{\parskip}{0pt}
			 \noindent\rule{.3\textwidth}{.5pt}\par
	 }
	 \ORIGchapterheadendvskip
}

\newcommand{\eclipse}{$\mbox{ECL}^i\mbox{PS}^e$}

%%%%%%%%%%%%%%%%%%%%%%%%%%%%%%%%%%%DOCUMENT%%%%%%%%%%%%%%%%%%%%%%%%%%%%%%%%%%%%%

\begin{document}

% Cover
	%\pagenumbering{alpha}
	\thispagestyle{empty}
	\includegraphics[height=1.2cm]{images/vsLogo.png}
	\hfill\includegraphics[height=1.2cm]{images/uniLogo.pdf}\\[1cm]

	\begin{center}
		\begin{LARGE}\bfseries Entwicklung eines intelligenten und kommunikationsfähigen Systems zur Überwachung elektrischer Größen 
					in autonomen Robotersystemen\end{LARGE}\\
		[2cm]
		\Large{Universität Kassel}\\
		[1cm]
		\large{Projektbericht von}\\
		\Large{Christian Schaub}\\
		[1cm]
		\large{At}\\ \Large{Distributed Systems Research Group}\\
		[1cm]
	\end{center}
	\begin{abstract}
Decision making is an important part of planning in multi-agent systems. Aside requirements like real-time capability, robustness against sensor noise and consistent decisions over all agents, reusability and intelligibility are often neglected. In the context of this project work a constraint vocabulary based on non-linear continuous constraint satisfaction problems is implemented. The design of the constraint vocabulary benefits its reusability and intelligibility, due to its parametrisability and hierarchical structure. The constraints of the vocabulary focus on the  Middle Size League of the RoboCup  domain. Due to the constraint vocabulary the positioning of agents of a MSL team is very simplified. The robustness against sensor noise is evaluated, considering the requirements of MSL~\cite{Aiello2007}.
	\end{abstract}

	\vfill
	\begin{Large}
		\begin{tabular}{r l}
			Reviewer: & Prof. Dr. Kurt Geihs\\
			[1cm]
			Supervisor: & Dipl.-Ing. Dominik Kirchner\\
		\end{tabular}
	\end{Large}
	\vfill

	\begin{center}Datum: Januar 21, 2012\end{center}

	

	\tableofcontents

	
	%% Hier können die einzelnen Kapitel inkludiert werden. Sie müssen in den 
% entsprechenden .TEX-Dateien vorliegen. Die Dateinamen können natürlich 
% angepasst werden.
%\include{Inhalt/Akronyme} %Akronyme.tex
\chapter{Related Work}
\label{cha:Related Work}

\section{Section 2}
\label{sec:2Section2}
Willkommen im Portal f"ur Elektronik, Maschinenbau und Mechatronik !
Dieses Portal soll euch beim lernen und diskutieren der einzelnen Studienf"acher behilflich sein, oder im Alltag als Knowledge-Base zur Verf"ugung stehen ! \\
F"ur jedes Studienfach wird in einem "Ubersichtsartikel der Inhalt zusammengefasst und die einzelnen Fachartikel in Beziehung zueinader gestellt. 
Kommen Formeln in den Fachartikeln vor, werden diese in einer Formelsammlung zu dem jeweiligen Studienfach zusammengef"uhrt. 
Am Ende soll jedes Studienfach eine"Ubersichtsartikel und wenn m"oglich eine Formelsammlung besitzen.

\subsection{Subcestion 2.1}
\label{subsec:2Subcestion2.1}

\begin{figure}[htb]
\centering
\includegraphics[width=0.2\textwidth]{mkDoc-Logo.png}
\caption{mkDoc}
\label{fig:mkDoc}
\end{figure}


\subsection{Subcestion 2.2}
\label{subsec:2Subcestion 2.2}
Willkommen im Portal f"ur Elektronik, Maschinenbau und Mechatronik !\footnote{\Vgl\Zitat[S.~11]{Sicherheitstechnik}}
Dieses Portal soll euch beim lernen und diskutieren der einzelnen Studienf"acher behilflich sein, oder im Alltag als Knowledge-Base zur Verf"ugung stehen ! \\
F"ur jedes Studienfach wird in einem "Ubersichtsartikel der Inhalt zusammengefasst und die einzelnen Fachartikel in Beziehung zueinader gestellt. 
Kommen Formeln in den Fachartikeln vor, werden diese in einer Formelsammlung zu dem jeweiligen Studienfach zusammengef"uhrt. 
Am Ende soll jedes Studienfach einen  "Ubersichtsartikel und wenn m"oglich eine Formelsammlung besitzen.

 %Einleitung.tex
\include{Inhalt/Relatedwork} %Vorgaben.tex
\include{Inhalt/Konzept} %Entwicklung.tex
\include{Inhalt/Umsetzung} %Umsetzung.tex
\include{Inhalt/Evaluierung} %Fazit.tex
\include{Inhalt/Zusammenfassung} %Zusammenfassung.tex

 %Inhalt.tex
\chapter{Einleitung}
\label{cha:Einleitung}
Woche 1: 
\begin{center}
\begin{tabular}{lllll}
Montag & Dienstag & Mittwoch & Donnerstag & Freitag\\
Einführung, 
Aufbau System, 
Erste Einarbeitung IDEs, 
Quartus Netbeans 
& 
Übertragung vhdl zum controller \newline, Einarbetung IDEs, Erste Unit Tests & micro IP Stack,, GUI auf Display einarbeitung- > einführung, DualCore kernel und Display Einführung, Tests erweitert & Probleme mit versch. Architekturen und Sprachen, Protokoll in Java impl. & Protokoll geändert 0xAAA als Start, Funktioniert, Erste GUI tests, Einarbeitung vcl files
\end{tabular}
\end{center}


\section{Related work}
\label{sec:1Related work}
bla

% ab Einleitung zählen !!
\pagenumbering{Roman}


\section{Funktionalitäten}
\label{sec:2Requirements}
\begin{enumerate}
 \item Kommunikation
 \item Messen versch. Spannungsbereiche 1,5,24,400 V
 \item Schalten von Lasten im Bereich > 1 A
 \item Intelligenz im C (Reflexe)
\end{enumerate}





%\begin{figure}[htb]
%\centering
%\includegraphics[width=0.2\textwidth]{mkDoc-Logo.png}
%\caption{mkDoc}
%\label{fig:mkDoc}
%\end{figure}

\chapter{Konzept}
\label{cha:Konzept}
\begin{flushleft}
Diskussion der Lösungen um Funktionalitäten zu erreichen... \linebreak
versch. Kanäle - redundant -> ausfallsicher \linebreak 
versch. Spannungsteiler, Z- Dioden, interne Schottky Dioden d. Controller \linebreak
kleine spannungen - OP Amp \linebreak
Implementierungnen Controller (Reflexe) \linebreak
Controller -- Transistor -- Relaisschaltung (Lasten)
\end{flushleft}


\chapter{Umsetzung}
\label{cha:Umsetzung}
Wie wurden konzeptionelle Lösungen umgesetzt, Beschreibung in Arbeitsschritten...
Observer -> System 

\section{Arbeitsschritte}
\label{sec:4Arbeitsschritte}

Darstellung der einzelnen Schritte

\section{Meilensteine}
\label{sec:5Meilensteine}

\chapter{Evaluierung}

\label{cha:Evaluierung}


Evaluierung der Funktionalitäten mittels verschiedener Versuchsaufbauten und anschliessender Dokumentation
der Messreihen...

\section{Messreihen}
\label{sec:6Messreihen}
\begin{center}
Messreihe 5 V Messungen:
\begin{tabular}{lllllll}
Spannungswert & Messung  0 & Messung 2 & Messung 3 & Messung 4 & Messung 5 & Messung 6\\
0 V & 0 & 0 & 0 & 0 & 0 & 0\\
1,3 V & 0 & 0 & 0 & 0 & 0 & 0\\
2,5 V & 0 & 0 & 0 & 0 & 0 & 0\\
3,5 V & 0 & 0 & 0 & 0 & 0 & 0\\
4,9 V & 0 & 0 & 0 & 0 & 0 & 0
\end{tabular}
\end{center}



\chapter{Zusammenfassung}
\label{cha:Zusammenfassung}
Zusammenfassung ....

\chapter{Abbildungsverzeichnis}
\label{cha:Abbildungsverzeichnis}


\bibliographystyle{plainnat}
\bibliography{bibtex_database}

\end{document}
